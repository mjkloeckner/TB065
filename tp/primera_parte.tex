
\section{Introducción}

En el presente trabajo se realiza un análisis en el dominio temporal de en
principio dos señales musicales de muestra, en las cuales se buscan porciones
cuasi-periódicas y no periódicas, y luego se filtran utilizando dos filtros
diferentes.  Por último, se generan mediante simulación de tres instrumentos
musicales diferentes, tres señales, las cuales se analizan y comparan las formas
de onda generadas.

\hypertarget{dominio-temporal}{%
\section{Dominio temporal}\label{dominio-temporal}}

\hypertarget{primer-muestra}{%
\subsection{Primer muestra}\label{primer-muestra}}

Para la primer muestra (archivo cancion1.wav) se realiza el
gráfico de la misma en el dominio temporal, el resultado se muestra en
la figura 1.

La frecuencia de muestreo de la misma es 44100 Hz, esto se obtiene del
mismo script utilizado para graficar el archivo, en el cual se divide la
cantidad de muestras por la duración del archivo.

\begin{figure}
\centering
\includegraphics{plot/cancion1.png}
\caption{Gráfico de archivo `cancion1.wav'}
\label{cancion1}
\end{figure}

\hypertarget{secciones-cuasi-periuxf3dicas}{%
\subsubsection{Secciones
cuasi-periódicas}\label{secciones-cuasi-periodicas}}

Cuando la señal tiene una estructura repetitiva, pero con variaciones en
amplitud, fase o frecuencia se dice que la señal es cuasi-periódica.

Realizando un análisis visual en detalle de la muestra se buscan partes
donde se comporte como tal, dos ejemplos se dan en las figuras 2 y 3. En
la primera se gráfica el intervalo \(0.248\) s a \(0.256\) s, mientras
que en la segunda se gráfica el intervalo \(0.520\) s a \(0.528\) s.

\begin{figure}
\centering
\includegraphics{plot/cancion1_0_248s_a_0_256s.png}
\caption{Sección cuasi-periódica archivo `cancion1.wav'}
\label{cancion1_seccion_cuasi_periodica}
\end{figure}

Dentro de los intervalos cuasi-periódicos graficados, se pueden detectar
visualmente los períodos fundamentales, los cuales se ven resaltados en
color celeste claro.

Curiosamente en ambos casos el período es igual y resulta \(T=0.003\) s,
lo cual corresponde con una frecuencia de aproximadamente \(333\) Hz.
Comparando con notas musicales de tabla esto se asemeja a una nota
\emph{E4}, la cual tiene una frecuencia de \(329.228\) Hz. Siendo que el
período se relaciona de manera inversa con la frecuencia y esta de
manera directa con la nota musical, se puede asegurar que al disminuir
este período la frecuencia aumentará y la nota musical será mas aguda,
mientras que en el caso contrario si aumenta el período la frecuencia
disminuye y también la nota musical.

\begin{figure}
\centering
\includegraphics{plot/cancion1_0_520s_a_0_528s.png}
\caption{Otra sección cuasi-periódica archivo `cancion1.wav'}
\label{cancion1_seccion_cuasi_periodica2}
\end{figure}

\hypertarget{segunda-muestra}{%
\subsection{Segunda muestra}\label{segunda-muestra}}

Utilizando el mismo script de python utilizado para la primer muestra
(archivo cancion1.wav) se gráfica la señal de la segunda
muestra (correspondiente al archivo cancion2.wav) en el dominio
temporal, en este caso se gráfica a partir del segundo 6 ya que antes de
esto la señal tiene amplitud nula, con lo cual no aporta información
significativa, el gráfico resultante se muestra en la figura 4.

La frecuencia fundamental de esta segunda muestra resulta \(48000\) Hz,
esto también se obtiene del script de python.

\begin{figure}
\centering
\includegraphics{plot/cancion2.png}
\caption{Gráfico de archivo `cancion2.wav'}
\label{cancion2}
\end{figure}

\hypertarget{secciones-no-periodicas}{%
\subsubsection{Secciones no-periódicas}\label{secciones-no-periodicas}}

A diferencia del análisis realizado sobre la primer muestra en busca de
secciones cuasi-periódicas, para esta segunda muestra se buscan
secciones no periódicas, esto es, secciones donde la señal no tiene un
patron repetitivo marcado. Se toman dos intervalos en los cuales la
señal de muestra se comporta como tal, el intervalo de \(14.72\) s a
\(14.73\) s y el intervalo \(26.57\) s a \(26.58\) s, ambos intervalos
se muestran graficados en las figura 5 y 6 respectivamente.

\begin{figure}
\centering
\includegraphics{plot/cancion2_14_72s_a_14_73s.png}
\caption{Sección no periódica archivo `cancion2.wav'}
\label{cancion2_seccion_no_periodica}
\end{figure}

\begin{figure}
\centering
\includegraphics{plot/cancion2_26_57s_a_26_58s.png}
\caption{Otra sección no periódica archivo `cancion2.wav'}
\label{cancion2_seccion_no_periodica2}
\end{figure}

Dado que las secciones son no periódicas, no se puede hablar de una
frecuencia fundamental como si se podía en las secciones
cuasi-periódicas en la primer muestra.





\hypertarget{filtrado}{%
\subsection{Filtrado}\label{filtrado}}

Para obtener la salida de la señal luego de pasarla por un filtro
(respuesta al impulso del primer filtro correspondiente al archivo
respuesta\_impulso\_1.txt y del segundo filtro correspondiente
al archivo respuesta\_impulso\_2.txt) es necesario realizar una
convolución entre la señal de entrada y la respuesta al impulso del
filtro, esto suponiendo que el filtro es un sistema LTI (si no lo fuera
no se podría calcular la salida solo teniendo la respuesta al impulso).

La salida del filtro 1 al aplicar la primer muestra se puede ver en la
figura \ref{cancion1_filter1_output_compare}, en este gráfico de la señal completa se observa que atenúa partes de la señal y
amplifica otras, en particular amplifica principalmente antes del
segundo \(6\) y atenúa drásticamente luego. En mayor detalle como se observa en la figura \ref{cancion1_filter1_output_compare_0_248_a_0_256} para el intervalo entre 0.248 y 0.256 segundos, se ve como actúa sobre la señal amplificando partes de la misma.

\begin{figure}[H]
\centering
\includegraphics{plot/cancion1_filter1_output_compare.png}
\caption{Primer muestra salida de filtro 1}
\label{cancion1_filter1_output_compare}
\end{figure}

\begin{figure}[H]
\centering
\includegraphics{plot/cancion1_filter1_output_compare_0_248_a_0_256.png}
\caption{Primer muestra salida de filtro 1 intervalo 0.248 a 0.256s}
\label{cancion1_filter1_output_compare_0_248_a_0_256}
\end{figure}

Aplicando el segundo filtro a la primer muestra resulta como se muestra
en el gráfico de figura \ref{cancion1_filter2_output_compare}. 
En esta figura de la señal completa Se puede ver que esta a diferencia del filtro 1, no atenúa o amplifica significativamente partes de la señal, si no
que realiza una leve atenuación de toda la señal. Analizando en detalle como se ve en la figura \ref{cancion1_filter2_output_compare_0_248_a_0_256} se puede ver el suavizado que realiza este filtro sobre la señal, esto sugiere que el filtro 2 es del tipo pasa bajos, ya que atenúa la frecuencias altas componentes en la señal original, haciendo que la señal resultante no tenga cambios abruptos.

\begin{figure}[H]
\centering
\includegraphics{plot/cancion1_filter2_output_compare.png}
\caption{Primer muestra salida de filtro 2}
\label{cancion1_filter2_output_compare}
\end{figure}

\begin{figure}[H]
\centering
\includegraphics{plot/cancion1_filter2_output_compare_0_248_a_0_256.png}
\caption{Primer muestra salida de filtro 2 intervalo 0.248 a 0.256s}
\label{cancion1_filter2_output_compare_0_248_a_0_256}
\end{figure}

De manera análoga para la segunda muestra se aplican los filtros mediante la
convolución entre la señal y la respuesta al impulso del respectivo filtro.

La salida de la segunda muestra al aplicar el primer filtro se puede ver en la
figura \ref{cancion2_filter1_output_compare} en la cual se gráfica la señal
completa y se aprecia como se atenúa la mayor parte, principalmente en la
primera mitad (antes del segundo 18.5 aproximadamente) y en menor medida en
la mitad restante, aunque en partes de la segunda mitad se atenúa drásticamente
de todas formas, como por ejemplo en el segundo 29 en el que se atenúa
aproximadamente un 70\% de la señal.

En la figura \ref{cancion2_6s_filter1_output_compare_26_57_a_26_58} se analiza
en mayor detalle la señal, en este caso el intervalo entre 26.57 y 26.58
segundos, en esta figura se observa como amplifica partes de la señal, en
particular, se observa que amplifica los picos en donde la señal tiene cambios
abruptos, es decir, donde la señal se compone de frecuencias altas, esto sugiere
que el primer filtro es del tipo pasa altos, es decir atenúa las frecuencias
bajas.

\begin{figure}
\centering
\includegraphics{plot/cancion2_6s_filter1_output_compare.png}
\caption{Segunda muestra salida de filtro 1}
\label{cancion2_filter1_output_compare}
\end{figure}

\begin{figure}
\centering
\includegraphics{plot/cancion2_6s_filter1_output_compare_26_57_a_26_58.png}
\caption{Segunda muestra salida de filtro 1 intervalo 26.57 a 26.58s}
\label{cancion2_6s_filter1_output_compare_26_57_a_26_58}
\end{figure}

\begin{figure}
\centering
\includegraphics{plot/cancion2_6s_filter2_output_compare.png}
\caption{Segunda muestra salida de filtro 2}
\label{cancion2_filter2_output_compare}
\end{figure}

\begin{figure}
\centering
\includegraphics{plot/cancion2_6s_filter2_output_compare_26_57_a_26_58.png}
\caption{Seguida muestra salida de filtro 2 intervalo 26.57 a 26.58s}
\label{cancion2_6s_filter2_output_compare_26_57_a_26_58}
\end{figure}


En las figura \ref{cancion2_filter2_output_compare} se muestra la señal completa
de la salida luego de aplicar el segundo filtro a la segunda muestra, de esta
figura no se pueden sacar grandes conclusiones mas que una leve atenuación de
toda la señal. Analizando en mayor detalle, por ejemplo el intervalo de 26.57 a
26.58 segundos, como se muestra en la figura
\ref{cancion2_6s_filter2_output_compare_26_57_a_26_58} se observa algo similar a
lo observado para la primer muestra luego de aplicar este filtro y es el
suavizado que realiza.

En todos los casos, tanto para la primer muestra como para la segunda y
tanto para el primer filtro como el segundo, escuchando la respectiva
salida se confirma lo analizado desde el punto de vista del gráfico de
la señal, pero ademas se aprecia que el primer filtro realiza una
atenuación de las frecuencias mas altas (sonidos agudos) mientras que el
segundo disminuye las frecuencias bajas (o sonidos graves) esto ultimo
no se aprecia en el gráfico de la señal, ya que parece no tener efecto
mas que une leve atenuación.

\hypertarget{sonido-de-diferentes-instrumentos}{%
\subsection{Sonido de diferentes
instrumentos}\label{sonido-de-diferentes-instrumentos}}

Se generaron tres muestras diferentes a las ya utilizadas, correspondientes con
la nota \emph{A4} (La4, \(440\) Hz) mediante la simulación de tres instrumentos
musicales distintos: un clarinete, una flauta y un violín. Las señales generadas
corresponden con los archivos a4\_clarinete.wav, a4\_flauta.wav y a4\_violin.wav
y los gráficos de cada señal se muestran en las figuras \ref{a4_clarinete_cutoff_time_comparison}, \ref{a4_clarinete_cutoff_time_comparison} y \ref{a4_violin_cutoff_time_comparison}
respectivamente.

Si bien todos los sonidos tienen la misma frecuencia, ya que es la misma
nota musical, el sonido escuchado percibido es diferente, esto puede ser
producto de la forma de onda generada por cada instrumento, lo cual
queda clara la diferencia en los respectivos gráficos.

Cada sonido percibido tiene características diferentes, el mas apagado o
neutro es el producido por la flauta, mientras que el mas agudo o
``afilado'' es el producido por el violín, el sonido del clarinete es un
intermedio entre ambos, un sonido ni muy agudo ni muy grave o apagado, y
con cierto carácter metálico.

Para el caso del clarinete, cuya señal se puede ver en la figura \ref{a4_clarinete_cutoff_time_comparison}, se puede ver que la onda se parece a una onda cuadrada. En el dominio de frecuencia, las ondas cuadradas ideales se componen de armónicos impares.

%\begin{figure}
%\centering
%\includegraphics{plot/a4_clarinete.png}
%\caption{Sonido de clarinete}
%\label{a4_clarinete}
%\end{figure}

Para la señal producida por la flauta que se puede ver en la figura \ref{a4_clarinete_cutoff_time_comparison},
se puede ver que se asemeja a una señal senoidal pura, aunque no tan
simétrica en los picos, las ondas sinodales en el dominio de frecuencia
tienen un único armónico, y es el fundamental, es por esto que el sonido
es mas neutro y no tan ``brillante'' o agudo dado que la frecuencia es
la misma en todos los casos.

%\begin{figure}
%\centering
%\includegraphics{plot/a4_flauta.png}
%\caption{Sonido de flauta}
%\label{a4_flauta}
%\end{figure}

Por ultimo para la señal producida por el violín, la cual se puede ver
en la figura \ref{a4_violin_cutoff_time_comparison}, se asemeja a una onda triangular con pendiente
decreciente, estas ondas triangulares en el dominio de frecuencia
también tienen armónicos impares como la onda cuadrada, pero estos
armónicos tienen mayor amplitud, es por esto que si bien tienen un
sonido similar, el sonido del violín es mas agudo.

%\begin{figure}[H]
%\centering
%\includegraphics{plot/a4_violin.png}
%\caption{Sonido de violín}
%\label{a4_violin}
%\end{figure}

